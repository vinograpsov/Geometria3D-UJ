\section{Macierze}


\subsection*{1. Oblicz wyznaczniki dla poniższych macierzy i jeśli to możliwe wyznacz macierze odwrotne.}

\begin{multline*}
    \begin{vmatrix}
    M_{xx} & M_{xy} & M_{xz} \\
    M_{yx} & M_{yy} & M_{yz} \\
    M_{zx} & M_{zy} & M_{zz}
    \end{vmatrix}
    =
    M_{xx}(-1)^{1+1}
    \begin{vmatrix}
    M_{yy} & M_{yz} \\
    M_{zy} & M_{zz}
    \end{vmatrix} \\
    + M_{xy}(-1)^{1+2}
    \begin{vmatrix}
    M_{yx} & M_{yz} \\
    M_{zx} & M_{zz}
    \end{vmatrix}
    + M_{xz}(-1)^{1+3}
    \begin{vmatrix}
    M_{yx} & M_{yy} \\
    M_{zx} & M_{zy}
    \end{vmatrix}
    \end{multline*}
    
    \begin{align*}
    &= M_{xx}M_{yy}M_{zz} - M_{xx}M_{yz}M_{zy} \\
    &\quad - M_{xy}M_{yx}M_{zz} + M_{xy}M_{yz}M_{zx} \\
    &\quad + M_{xz}M_{yx}M_{zy} - M_{xz}M_{yy}M_{zx}
    \end{align*}


\begin{enumerate}
    \item[(a)] \( M = \begin{bmatrix} 2 & 5 \\ 2 & 7 \end{bmatrix} \)
    
    \item[(b)] \( M = \begin{bmatrix} 1 & 8 & 9 \\ 0 & 4 & 3 \\ 5 & 0 & 5 \end{bmatrix} \)
    
    \item[(c)] \( M = \begin{bmatrix} 1 & 6 & 4 \\ 2 & 7 & 3 \\ 8 & 9 & 5  \end{bmatrix} \)
    
    \item[(d)] \( M = \begin{bmatrix} 1 & 8 & 9 & 2 \\ \\ 0 & 2 & 4 & 3 \\ 5 & 0 & 2 & 5 \\ 5 & 1 & 4 & 2 \end{bmatrix} \)
    
    \item[(e)] \( M = \begin{bmatrix} 1 & 6 & 4 & 0 \\ 2 & 7 & 3 & 0 \\ 0 & 0 & 0 & 0 \\ 8 & 9 & 0 & 5 \end{bmatrix} \)
\end{enumerate}

\subsection*{2. Wyznacz macierze odwrotne używając metody Gaussa-Jordana.}

\begin{enumerate}
    \item[(a)] \( M = \begin{bmatrix} 1 & 1 & 2 \\ 2 & 3 & 4 \\ 2 & 1 & 5 \end{bmatrix} \)
    
    \item[(b)] \( M = \begin{bmatrix} 1 & 6 & 4 \\ 2 & 7 & 3 \\ 8 & 9 & 5 \end{bmatrix} \)
\end{enumerate}

\subsection*{3. Dokonaj diagonalizacji macierzy.}

\begin{enumerate}
    \item[(a)] \( M = \begin{bmatrix} 1 & 2 & 3 \\ 2 & 3 & 4 \\ 2 & 1 & 5 \end{bmatrix} \)
    
    \item[(b)] \( M = \begin{bmatrix} 1 & 6 & 4 \\ 2 & 7 & 3 \\ 8 & 9 & 5 \end{bmatrix} \)
\end{enumerate}

\subsection*{4. Znajdź wartości własne i wektory własne macierzy.}

\begin{enumerate}
    \item[(a)] \( M = \begin{bmatrix} 1 & 0.5 \\ 0 & 2 \end{bmatrix} \)
    
    \item[(b)] \( M = \begin{bmatrix} 5 & 3 \\ -6 & -4 \end{bmatrix} \)
    
    \item[(c)] \( M = \begin{bmatrix} 1 & 1 \\ 4 & 1 \end{bmatrix} \)
    
    \item[(d)] \( M = \begin{bmatrix} 2 & 1 & 0 \\ 1 & 2 & 0 \\ 0 & 1 & 1 \end{bmatrix} \)
    
    \item[(e)] \( M = \begin{bmatrix} 2 & 1 & 0 \\ -6 & 1 & -6 \\ -3 & 1 & -1 \end{bmatrix} \)
    
    \item[(f)] \( M = \begin{bmatrix} 3 & 0 & 0 \\ -1 & 0 & 4 \\ 2 & 1 & 0 \end{bmatrix} \)
\end{enumerate}

\subsection*{5. Wykonaj operacje skalowania macierzą \( M \) dla macierzy wierzchołków \( N \).}


\noindent
Macierz skalowania w trójwymiarowej przestrzeni można zapisać jako:
\[
S(\alpha,\beta,\gamma) 
\;=\;
\begin{pmatrix}
\alpha & 0      & 0 \\
0      & \beta  & 0 \\
0      & 0      & \gamma
\end{pmatrix},
\]
gdzie \(\alpha,\beta,\gamma\) to dodatnie liczby rzeczywiste.

\bigskip
\noindent
Działanie tej transformacji na wektor \(\mathbf{V} = (V_x, V_y, V_z)^\top\) daje wektor
\[
\mathbf{V}' \;=\; S(\alpha,\beta,\gamma)\,\mathbf{V}
\;=\;
\begin{pmatrix}
\alpha\,V_x \\
\beta\,V_y \\
\gamma\,V_z
\end{pmatrix}.
\]


\begin{enumerate}
        \item[(a)] 
        \[
        M = \begin{bmatrix} 
        1 & 0 & 0 \\ 
        0 & 2 & 0 \\ 
        0 & 0 & 3 
        \end{bmatrix}, 
        \quad
        N = \begin{bmatrix} 
        1 & 3 \\ 
        2 & 1 \\ 
        -1 & 2 
        \end{bmatrix}
        \]
    
        \item[(b)] 
        \[
        M = \begin{bmatrix} 
        1 & 0 & 0 \\ 
        0 & 2 & 0 \\ 
        0 & 0 & 3 
        \end{bmatrix}, 
        \quad
        N = \begin{bmatrix} 
        1 & -2 & 5 \\ 
        -2 & 2 & 3 \\ 
        3 & 2 & 4 
        \end{bmatrix}
        \]
    
        \item[(c)] 
        \[
        M = \begin{bmatrix} 
        3 & 0 & 0 \\ 
        0 & 4 & 0 \\ 
        0 & 0 & 1 
        \end{bmatrix}, 
        \quad
        N = \begin{bmatrix} 
        0 & 1 & 2 & -4 \\ 
        0 & 1 & -1 & 7 \\ 
        0 & 1 & 3 & 5 
        \end{bmatrix}
        \]
\end{enumerate}

\subsection*{6. Zbuduj macierz skalowania \( S \) dla której wynikiem operacji na punkcie \( P_1 \) będzie punkt \( P_2 \).}

\begin{enumerate}
    \item[(a)] \( P_1 = \begin{bmatrix} 0 \\ 2 \\ 4 \end{bmatrix}, \quad
    P_2 = \begin{bmatrix} 0 \\ 4 \\ 2 \end{bmatrix} \)
\end{enumerate}

\subsection*{7. Siatka geometryczna obiektu zawiera punkty \( P_1, P_2, P_3, P_4 \). Oblicz pozycję punktów \( P_1', P_2', P_3', P_4' \), które powstaną po zastosowaniu do nich macierzy skalowania \( S \) względem punktu \( P_T \).}

\noindent
Przy skalowaniu punktu $\mathbf{P}$ względem tzw.\ pivotu (punktu odniesienia) $P_T$
z~użyciem macierzy skalowania $S$ (np.\ $3\times 3$), nowa pozycja $\mathbf{P}'$ wyznaczana jest ze wzoru:
\[
\mathbf{P}' 
\;=\;
\mathbf{P}_T \;+\; S\,\bigl(\mathbf{P} - \mathbf{P}_T\bigr).
\]

\medskip
\noindent
\textbf{Algorytm (krok po kroku):}
\begin{enumerate}
    \item \emph{Przesunięcie pivotu do początku:} oblicz $\mathbf{P} - \mathbf{P}_T$.
    \item \emph{Skalowanie:} pomnóż otrzymany wektor przez macierz $S$, czyli 
          $S \bigl(\mathbf{P} - \mathbf{P}_T\bigr)$.
    \item \emph{Przesunięcie z powrotem:} dodaj $\mathbf{P}_T$,
          otrzymując $\mathbf{P}' = \mathbf{P}_T + S\bigl(\mathbf{P} - \mathbf{P}_T\bigr)$.
\end{enumerate}


\begin{enumerate}
    \item[(a)] 
    \[
    P_1 = \begin{bmatrix} 0 \\ 0 \\ 0 \end{bmatrix}, 
    \quad P_2 = \begin{bmatrix} 1 \\ 1 \\ 0 \end{bmatrix}, 
    \quad P_3 = \begin{bmatrix} -1 \\ 1 \\ 0 \end{bmatrix}, 
    \quad P_4 = \begin{bmatrix} 2 \\ 2 \\ 0 \end{bmatrix}
    \]
    
    \[
    S = \begin{bmatrix} 2 & 0 & 0 \\ 0 & 2 & 0 \\ 0 & 0 & 1 \end{bmatrix}, 
    \quad P_T = \begin{bmatrix} 1 \\ 1 \\ 0 \end{bmatrix}
    \]

    \item[(b)] 
    \[
    P_1 = \begin{bmatrix} 1 \\ 2 \\ 3 \end{bmatrix}, 
    \quad P_2 = \begin{bmatrix} 4 \\ -1 \\ 9 \end{bmatrix}, 
    \quad P_3 = \begin{bmatrix} 0 \\ 0 \\ 0 \end{bmatrix}, 
    \quad P_4 = \begin{bmatrix} -4 \\ -2 \\ -1 \end{bmatrix}
    \]
    
    \[
    S = \begin{bmatrix} 2 & 0 & 0 \\ 0 & -3 & 0 \\ 0 & 0 & 4 \end{bmatrix}, 
    \quad P_T = \begin{bmatrix} -4 \\ -2 \\ -1 \end{bmatrix}
    \]

    \item[(c)] 
    \[
    P_1 = \begin{bmatrix} 0 \\ 0 \\ 0 \end{bmatrix}, 
    \quad P_2 = \begin{bmatrix} 5 \\ -5 \\ 5 \end{bmatrix}, 
    \quad P_3 = \begin{bmatrix} 4 \\ 1 \\ 0 \end{bmatrix}, 
    \quad P_4 = \begin{bmatrix} -4 \\ -2 \\ -1 \end{bmatrix}
    \]
    
    \[
    S = \begin{bmatrix} 1 & 0 & 0 \\ 0 & 2 & 0 \\ 0 & 0 & 3 \end{bmatrix}, 
    \quad P_T = \begin{bmatrix} 3 \\ 4 \\ 1 \end{bmatrix}
    \]
\end{enumerate}

\subsection*{8. Wyznacz wektor \( \mathbf{V'} \) będący wektorem powstałym w wyniku obrotu wektora \( \mathbf{V} \) o kąt \( \alpha \) w osi \( \mathbf{Z} \).}



\[
R_x(\theta) \;=\;
\begin{pmatrix}
1 & 0 & 0 \\
0 & \cos\theta & -\sin\theta \\
0 & \sin\theta & \cos\theta
\end{pmatrix},
\quad
R_y(\theta) \;=\;
\begin{pmatrix}
\cos\theta & 0 & \sin\theta \\
0 & 1 & 0 \\
-\sin\theta & 0 & \cos\theta
\end{pmatrix},
\quad
R_z(\theta) \;=\;
\begin{pmatrix}
\cos\theta & -\sin\theta & 0 \\
\sin\theta & \cos\theta & 0 \\
0 & 0 & 1
\end{pmatrix}.
\]

\begin{table}[h!]
    \centering
    \begin{tabular}{c|c|c|c}
    \hline
    \textbf{Kąt} & \textbf{Kąt w radianach} & \(\sin(\theta)\) & \(\cos(\theta)\) \\
    \hline
    \(0^\circ\)   & \(0\)                   & \(0\)                  & \(1\)                   \\
    \(30^\circ\)  & \(\frac{\pi}{6}\)       & \(\frac{1}{2}\)        & \(\frac{\sqrt{3}}{2}\)  \\
    \(45^\circ\)  & \(\frac{\pi}{4}\)       & \(\frac{\sqrt{2}}{2}\) & \(\frac{\sqrt{2}}{2}\)  \\
    \(60^\circ\)  & \(\frac{\pi}{3}\)       & \(\frac{\sqrt{3}}{2}\) & \(\frac{1}{2}\)         \\
    \(90^\circ\)  & \(\frac{\pi}{2}\)       & \(1\)                  & \(0\)                   \\
    \(120^\circ\) & \(\frac{2\pi}{3}\)      & \(\frac{\sqrt{3}}{2}\) & \(-\tfrac{1}{2}\)       \\
    \(135^\circ\) & \(\frac{3\pi}{4}\)      & \(\tfrac{\sqrt{2}}{2}\)& \(-\tfrac{\sqrt{2}}{2}\)\\
    \(150^\circ\) & \(\frac{5\pi}{6}\)      & \(\tfrac{1}{2}\)       & \(-\tfrac{\sqrt{3}}{2}\)\\
    \(180^\circ\) & \(\pi\)                 & \(0\)                  & \(-1\)                  \\
    \hline
    \end{tabular}
    \caption{Podstawowe wartości \(\sin\theta\) i~\(\cos\theta\) dla najważniejszych kątów (w~stopniach i~radianach).}
    \label{tab:sin-cos-extended}
    \end{table}


\begin{enumerate}
    \item[(a)] 
    \[
    \mathbf{V} = \begin{bmatrix} 9 \\ 3 \\ 0 \end{bmatrix}, \quad \alpha = 60^\circ
    \]

    \item[(b)] 
    \[
    \mathbf{V} = \begin{bmatrix} 2 \\ 2 \\ 0 \end{bmatrix}, \quad \alpha = 45^\circ
    \]

    \item[(c)] 
    \[
    \mathbf{V} = \begin{bmatrix} 1 \\ 0 \\ 0 \end{bmatrix}, \quad \alpha = 30^\circ
    \]

    \item[(d)] 
    \[
    \mathbf{V} = \begin{bmatrix} 1 \\ 1 \\ 0 \end{bmatrix}, \quad \alpha = 90^\circ
    \]
\end{enumerate}



\subsection*{9. Zbuduj macierz obrotu \( \mathbf{R} \) pozwalającą wykonać operację obrotu o kąt \( \alpha \) wokół osi \( \mathbf{Z} \).}

Macierz obrotu wokół osi \( Z \) ma postać:


\begin{enumerate}
    \item[(a)] \( \alpha = 30^\circ \)
    \item[(b)] \( \alpha = 45^\circ \)
    \item[(c)] \( \alpha = 60^\circ \)
    \item[(d)] \( \alpha = 90^\circ \)
\end{enumerate}







\subsection*{10. Pracując we współrzędnych jednorodnych zbuduj macierz transformacji \( \mathbf{F} \) pozwalającą wykonać operację skalowania \( \mathbf{S} \) oraz translacji \( \mathbf{T} \) jednocześnie. 
Oblicz pozycję punktów \( P_1', P_2', P_3' \), które powstaną po zastosowaniu do nich macierzy transformacji \( \mathbf{F} \).}


\noindent
W~układzie współrzędnych jednorodnych trójwymiarowy punkt 
\[
\mathbf{P} 
\;=\; 
(x,\;y,\;z)
\]
zapisujemy jako wektor 4D:
\[
\widetilde{\mathbf{P}}
\;=\;
\begin{pmatrix}
x\\[3pt]
y\\[3pt]
z\\[3pt]
1
\end{pmatrix}.
\]

\medskip

\noindent
Jeśli chcemy wykonać \emph{jednocześnie} skalowanie o~współczynniki 
\(S_x, S_y, S_z\) oraz translację o~wektor \(\mathbf{T} = (T_x, T_y, T_z)\),
to stosujemy macierz \(\mathbf{F} \in \mathbb{R}^{4\times 4}\) w~postaci:
\[
\mathbf{F}
\;=\;
\begin{pmatrix}
S_x & 0   & 0   & T_x \\[4pt]
0   & S_y & 0   & T_y \\[4pt]
0   & 0   & S_z & T_z \\[4pt]
0   & 0   & 0   & 1
\end{pmatrix}.
\]

\medskip

\noindent
Wtedy \emph{nowy} punkt 
\(\mathbf{P}' = (x',\,y',\,z')\)
otrzymujemy przez wymnożenie:
\[
\widetilde{\mathbf{P}}'
\;=\;
\mathbf{F}\,\widetilde{\mathbf{P}}
\;\quad\Longrightarrow\quad
\mathbf{P}'
\;=\;
\bigl(x',\,y',\,z'\bigr).
\]
Dokładniej:
\[
\underbrace{
\begin{pmatrix}
x'\\[3pt]
y'\\[3pt]
z'\\[3pt]
1
\end{pmatrix}
}_{\widetilde{\mathbf{P}}'}
\;=\;
\underbrace{
\begin{pmatrix}
S_x & 0   & 0   & T_x \\[4pt]
0   & S_y & 0   & T_y \\[4pt]
0   & 0   & S_z & T_z \\[4pt]
0   & 0   & 0   & 1
\end{pmatrix}
}_{\mathbf{F}}
\cdot
\underbrace{
\begin{pmatrix}
x\\[2pt]
y\\[2pt]
z\\[2pt]
1
\end{pmatrix}
}_{\widetilde{\mathbf{P}}}
=
\begin{pmatrix}
S_x\,x + T_x\\[3pt]
S_y\,y + T_y\\[3pt]
S_z\,z + T_z\\[3pt]
1
\end{pmatrix}.
\]
Widać, że \(\mathbf{P}'\) to najpierw \emph{skalowanie} współrzędnych 
\((x,y,z)\)
wzdłuż każdej osi, a~następnie \emph{przesunięcie} o~\(\bigl(T_x, T_y, T_z\bigr)\).


\begin{enumerate}
    \item[(a)] 
    \[
    P_1 = \begin{bmatrix} 0 \\ 0 \\ 0  \end{bmatrix}, 
    \quad P_2 = \begin{bmatrix} 1 \\ 1 \\ 1 \end{bmatrix}, 
    \quad P_3 = \begin{bmatrix} 2 \\ 5 \\ 1  \end{bmatrix}
    \]

    \[
    S = \begin{bmatrix} 2 & 0 & 0 \\ 0 & 2 & 0 \\ 0 & 0 & 2 \end{bmatrix}, 
    \quad T = \begin{bmatrix} 0 \\ 0 \\ 0 \end{bmatrix}
    \]

    \item[(b)] 
    \[
    P_1 = \begin{bmatrix} 4 \\ 2 \\ 0  \end{bmatrix}, 
    \quad P_2 = \begin{bmatrix} 1 \\ 2 \\ 3  \end{bmatrix}, 
    \quad P_3 = \begin{bmatrix} 2 \\ 5 \\ 1  \end{bmatrix}
    \]

    \[
    S = \begin{bmatrix} 1 & 0 & 0 \\ 0 & 1 & 0 \\ 0 & 0 & 1 \end{bmatrix}, 
    \quad T = \begin{bmatrix} 1 \\ 1 \\ 1 \end{bmatrix}
    \]

    \item[(c)] 
    \[
    P_1 = \begin{bmatrix} 0 \\ 0 \\ 0  \end{bmatrix}, 
    \quad P_2 = \begin{bmatrix} 1 \\ 1 \\ 1  \end{bmatrix}, 
    \quad P_3 = \begin{bmatrix} 2 \\ 5 \\ 1  \end{bmatrix}
    \]

    \[
    S = \begin{bmatrix} 0 & 0 & 0 \\ 0 & 0 & 0 \\ 0 & 0 & 0 \end{bmatrix}, 
    \quad T = \begin{bmatrix} 1 \\ 1 \\ 1 \end{bmatrix}
    \]

    \item[(f)] 
    \[
    P_1 = \begin{bmatrix} 0 \\ 0 \\ 0  \end{bmatrix}, 
    \quad P_2 = \begin{bmatrix} 1 \\ 1 \\ 1  \end{bmatrix}, 
    \quad P_3 = \begin{bmatrix} 2 \\ 5 \\ 1  \end{bmatrix}
    \]

    \[
    S_x = 2, \quad S_y = 1, \quad S_z = 3, 
    \quad T = \begin{bmatrix} 1 \\ 1 \\ 2 \end{bmatrix}
    \]

\end{enumerate}


\subsection*{11. Zbuduj macierz transformacji \( \mathbf{M} \) pozwalającą wykonać kolejno operację obrotu o kąt \( \alpha \) wokół osi \( \mathbf{Z} \), obrotu o kąt \( \beta \) wokół osi \( \mathbf{Y} \) jednocześnie. 
Oblicz pozycje punktów \( P_1', P_2', P_3' \), które powstaną po zastosowaniu macierzy transformacji \( \mathbf{M} \) do punktów \( P_1, P_2, P_3 \).}

\begin{enumerate}
    \item[(a)] 
    \[
    P_1 = \begin{bmatrix} 0 \\ 0 \\ 0 \end{bmatrix}, 
    \quad P_2 = \begin{bmatrix} 1 \\ 1 \\ 1 \end{bmatrix}, 
    \quad P_3 = \begin{bmatrix} 2 \\ 5 \\ 1 \end{bmatrix}
    \]
    
    \[
    \quad \alpha = 45^\circ, \quad \beta = 90^\circ
    \]
\end{enumerate}

\subsection*{12. Pracując we współrzędnych jednorodnych zbuduj macierz transformacji \( \mathbf{F} \) 
pozwalającą wykonać kolejno operację skalowania \( \mathbf{S} \), obrotu o kąt \( \alpha \) wokół osi \( \mathbf{X} \), 
translacji \( \mathbf{T_1} \), obrotu o kąt \( \beta \) wokół osi \( \mathbf{Y} \) i translacji \( \mathbf{T_2} \) jednocześnie.
Oblicz pozycję punktów \( P_1', P_2', P_3' \), które powstaną po zastosowaniu macierzy transformacji \( \mathbf{F} \) 
do punktów \( P_1, P_2, P_3 \).}

\begin{enumerate}
    \item[(a)] 
    \[
    P_1 = \begin{bmatrix} 0 \\ 0 \\ 0 \end{bmatrix}, 
    \quad P_2 = \begin{bmatrix} 1 \\ 1 \\ 1 \end{bmatrix}, 
    \quad P_3 = \begin{bmatrix} 2 \\ 5 \\ 1 \end{bmatrix}
    \]

    \[
    S_x = 2, \quad S_y = 1, \quad S_z = 2, 
    \quad T_1 = \begin{bmatrix} 1 \\ 1 \\ 2 \end{bmatrix}, 
    \quad T_2 = \begin{bmatrix} 0 \\ 1 \\ 2 \end{bmatrix}
    \]

    \[
    \alpha = 45^\circ, \quad \beta = 90^\circ
    \]

    \item[(b)] 
    \[
    P_1 = \begin{bmatrix} 0 \\ 2 \\ 4 \end{bmatrix}, 
    \quad P_2 = \begin{bmatrix} 3 \\ 2 \\ 1 \end{bmatrix}, 
    \quad P_3 = \begin{bmatrix} 3 \\ -2 \\ 5 \end{bmatrix}
    \]

    \[
    S_x = 4, \quad S_y = 2, \quad S_z = 2, 
    \quad T_1 = \begin{bmatrix} 1 \\ -2 \\ 3 \end{bmatrix}, 
    \quad T_2 = \begin{bmatrix} 2 \\ 1 \\ 0 \end{bmatrix}
    \]

    \[
    \alpha = 45^\circ, \quad \beta = 120^\circ
    \]

    \item[(c)] 
    \[
    P_1 = \begin{bmatrix} 4 \\ 2 \\ 0 \end{bmatrix}, 
    \quad P_2 = \begin{bmatrix} 1 \\ 2 \\ 3 \end{bmatrix}, 
    \quad P_3 = \begin{bmatrix} -1 \\ 2 \\ -4 \end{bmatrix}
    \]

    \[
    S_x = 0.5, \quad S_y = 2, \quad S_z = 3, 
    \quad T_1 = \begin{bmatrix} 5 \\ -1 \\ 2 \end{bmatrix}, 
    \quad T_2 = \begin{bmatrix} 2 \\ 0 \\ 4 \end{bmatrix}
    \]

    \[
    \alpha = 150^\circ, \quad \beta = 270^\circ
    \]
\end{enumerate}

