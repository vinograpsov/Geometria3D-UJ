\section{Macierze}


\subsection*{1. Oblicz wyznaczniki dla poniższych macierzy i jeśli to możliwe wyznacz macierze odwrotne.}

\begin{multline*}
    \begin{vmatrix}
    M_{xx} & M_{xy} & M_{xz} \\
    M_{yx} & M_{yy} & M_{yz} \\
    M_{zx} & M_{zy} & M_{zz}
    \end{vmatrix}
    =
    M_{xx}(-1)^{1+1}
    \begin{vmatrix}
    M_{yy} & M_{yz} \\
    M_{zy} & M_{zz}
    \end{vmatrix} \\
    + M_{xy}(-1)^{1+2}
    \begin{vmatrix}
    M_{yx} & M_{yz} \\
    M_{zx} & M_{zz}
    \end{vmatrix}
    + M_{xz}(-1)^{1+3}
    \begin{vmatrix}
    M_{yx} & M_{yy} \\
    M_{zx} & M_{zy}
    \end{vmatrix}
    \end{multline*}
    
    \begin{align*}
    &= M_{xx}M_{yy}M_{zz} - M_{xx}M_{yz}M_{zy} \\
    &\quad - M_{xy}M_{yx}M_{zz} + M_{xy}M_{yz}M_{zx} \\
    &\quad + M_{xz}M_{yx}M_{zy} - M_{xz}M_{yy}M_{zx}
    \end{align*}


\begin{enumerate}
    \item[(a)] \( M = \begin{bmatrix} 2 & 5 \\ 2 & 7 \end{bmatrix} \)
    
    \item[(b)] \( M = \begin{bmatrix} 1 & 8 & 9 \\ 0 & 4 & 3 \\ 5 & 0 & 5 \end{bmatrix} \)
    
    \item[(c)] \( M = \begin{bmatrix} 1 & 6 & 4 \\ 2 & 7 & 3 \\ 8 & 9 & 5  \end{bmatrix} \)
    
    \item[(d)] \( M = \begin{bmatrix} 1 & 8 & 9 & 2 \\ \\ 0 & 2 & 4 & 3 \\ 5 & 0 & 2 & 5 \\ 5 & 1 & 4 & 2 \end{bmatrix} \)
    
    \item[(e)] \( M = \begin{bmatrix} 1 & 6 & 4 & 0 \\ 2 & 7 & 3 & 0 \\ 0 & 0 & 0 & 0 \\ 8 & 9 & 0 & 5 \end{bmatrix} \)
\end{enumerate}

\subsection*{2. Wyznacz macierze odwrotne używając metody Gaussa-Jordana.}

\begin{enumerate}
    \item[(a)] \( M = \begin{bmatrix} 1 & 1 & 2 \\ 2 & 3 & 4 \\ 2 & 1 & 5 \end{bmatrix} \)
    
    \item[(b)] \( M = \begin{bmatrix} 1 & 6 & 4 \\ 2 & 7 & 3 \\ 8 & 9 & 5 \end{bmatrix} \)
\end{enumerate}

\subsection*{3. Dokonaj diagonalizacji macierzy.}

\begin{enumerate}
    \item[(a)] \( M = \begin{bmatrix} 1 & 2 & 3 \\ 2 & 3 & 4 \\ 2 & 1 & 5 \end{bmatrix} \)
    
    \item[(b)] \( M = \begin{bmatrix} 1 & 6 & 4 \\ 2 & 7 & 3 \\ 8 & 9 & 5 \end{bmatrix} \)
\end{enumerate}

\subsection*{4. Znajdź wartości własne i wektory własne macierzy.}

\begin{enumerate}
    \item[(a)] \( M = \begin{bmatrix} 1 & 0.5 \\ 0 & 2 \end{bmatrix} \)
    
    \item[(b)] \( M = \begin{bmatrix} 5 & 3 \\ -6 & -4 \end{bmatrix} \)
    
    \item[(c)] \( M = \begin{bmatrix} 1 & 1 \\ 4 & 1 \end{bmatrix} \)
    
    \item[(d)] \( M = \begin{bmatrix} 2 & 1 & 0 \\ 1 & 2 & 0 \\ 0 & 1 & 1 \end{bmatrix} \)
    
    \item[(e)] \( M = \begin{bmatrix} 2 & 1 & 0 \\ -6 & 1 & -6 \\ -3 & 1 & -1 \end{bmatrix} \)
    
    \item[(f)] \( M = \begin{bmatrix} 3 & 0 & 0 \\ -1 & 0 & 4 \\ 2 & 1 & 0 \end{bmatrix} \)
\end{enumerate}

\subsection*{5. Wykonaj operacje skalowania macierzą \( M \) dla macierzy wierzchołków \( N \).}


\noindent
Macierz skalowania w trójwymiarowej przestrzeni można zapisać jako:
\[
S(\alpha,\beta,\gamma) 
\;=\;
\begin{pmatrix}
\alpha & 0      & 0 \\
0      & \beta  & 0 \\
0      & 0      & \gamma
\end{pmatrix},
\]
gdzie \(\alpha,\beta,\gamma\) to dodatnie liczby rzeczywiste.

\bigskip
\noindent
Działanie tej transformacji na wektor \(\mathbf{V} = (V_x, V_y, V_z)^\top\) daje wektor
\[
\mathbf{V}' \;=\; S(\alpha,\beta,\gamma)\,\mathbf{V}
\;=\;
\begin{pmatrix}
\alpha\,V_x \\
\beta\,V_y \\
\gamma\,V_z
\end{pmatrix}.
\]


\begin{enumerate}
        \item[(a)] 
        \[
        M = \begin{bmatrix} 
        1 & 0 & 0 \\ 
        0 & 2 & 0 \\ 
        0 & 0 & 3 
        \end{bmatrix}, 
        \quad
        N = \begin{bmatrix} 
        1 & 3 \\ 
        2 & 1 \\ 
        -1 & 2 
        \end{bmatrix}
        \]
    
        \item[(b)] 
        \[
        M = \begin{bmatrix} 
        1 & 0 & 0 \\ 
        0 & 2 & 0 \\ 
        0 & 0 & 3 
        \end{bmatrix}, 
        \quad
        N = \begin{bmatrix} 
        1 & -2 & 5 \\ 
        -2 & 2 & 3 \\ 
        3 & 2 & 4 
        \end{bmatrix}
        \]
    
        \item[(c)] 
        \[
        M = \begin{bmatrix} 
        3 & 0 & 0 \\ 
        0 & 4 & 0 \\ 
        0 & 0 & 1 
        \end{bmatrix}, 
        \quad
        N = \begin{bmatrix} 
        0 & 1 & 2 & -4 \\ 
        0 & 1 & -1 & 7 \\ 
        0 & 1 & 3 & 5 
        \end{bmatrix}
        \]
\end{enumerate}

\subsection*{6. Zbuduj macierz skalowania \( S \) dla której wynikiem operacji na punkcie \( P_1 \) będzie punkt \( P_2 \).}

\begin{enumerate}
    \item[(a)] \( P_1 = \begin{bmatrix} 0 \\ 2 \\ 4 \end{bmatrix}, \quad
    P_2 = \begin{bmatrix} 0 \\ 4 \\ 2 \end{bmatrix} \)
\end{enumerate}

\subsection*{7. Siatka geometryczna obiektu zawiera punkty \( P_1, P_2, P_3, P_4 \). Oblicz pozycję punktów \( P_1', P_2', P_3', P_4' \), które powstaną po zastosowaniu do nich macierzy skalowania \( S \) względem punktu \( P_T \).}

\begin{enumerate}
    \item[(a)] 
    \[
    P_1 = \begin{bmatrix} 0 \\ 0 \\ 0 \end{bmatrix}, 
    \quad P_2 = \begin{bmatrix} 1 \\ 1 \\ 0 \end{bmatrix}, 
    \quad P_3 = \begin{bmatrix} -1 \\ 1 \\ 0 \end{bmatrix}, 
    \quad P_4 = \begin{bmatrix} 2 \\ 2 \\ 0 \end{bmatrix}
    \]
    
    \[
    S = \begin{bmatrix} 2 & 0 & 0 \\ 0 & 2 & 0 \\ 0 & 0 & 1 \end{bmatrix}, 
    \quad P_T = \begin{bmatrix} 1 \\ 1 \\ 0 \end{bmatrix}
    \]

    \item[(b)] 
    \[
    P_1 = \begin{bmatrix} 1 \\ 2 \\ 3 \end{bmatrix}, 
    \quad P_2 = \begin{bmatrix} 4 \\ -1 \\ 9 \end{bmatrix}, 
    \quad P_3 = \begin{bmatrix} 0 \\ 0 \\ 0 \end{bmatrix}, 
    \quad P_4 = \begin{bmatrix} -4 \\ -2 \\ -1 \end{bmatrix}
    \]
    
    \[
    S = \begin{bmatrix} 2 & 0 & 0 \\ 0 & -3 & 0 \\ 0 & 0 & 4 \end{bmatrix}, 
    \quad P_T = \begin{bmatrix} -4 \\ -2 \\ -1 \end{bmatrix}
    \]

    \item[(c)] 
    \[
    P_1 = \begin{bmatrix} 0 \\ 0 \\ 0 \end{bmatrix}, 
    \quad P_2 = \begin{bmatrix} 5 \\ -5 \\ 5 \end{bmatrix}, 
    \quad P_3 = \begin{bmatrix} 4 \\ 1 \\ 0 \end{bmatrix}, 
    \quad P_4 = \begin{bmatrix} -4 \\ -2 \\ -1 \end{bmatrix}
    \]
    
    \[
    S = \begin{bmatrix} 1 & 0 & 0 \\ 0 & 2 & 0 \\ 0 & 0 & 3 \end{bmatrix}, 
    \quad P_T = \begin{bmatrix} 3 \\ 4 \\ 1 \end{bmatrix}
    \]
\end{enumerate}

\subsection*{8. Wyznacz wektor \( \mathbf{V'} \) będący wektorem powstałym w wyniku obrotu wektora \( \mathbf{V} \) o kąt \( \alpha \) w osi \( \mathbf{Z} \).}

\begin{enumerate}
    \item[(a)] 
    \[
    \mathbf{V} = \begin{bmatrix} 9 \\ 3 \\ 0 \end{bmatrix}, \quad \alpha = 60^\circ
    \]

    \item[(b)] 
    \[
    \mathbf{V} = \begin{bmatrix} 2 \\ 2 \\ 0 \end{bmatrix}, \quad \alpha = 45^\circ
    \]

    \item[(c)] 
    \[
    \mathbf{V} = \begin{bmatrix} 1 \\ 0 \\ 0 \end{bmatrix}, \quad \alpha = 30^\circ
    \]

    \item[(d)] 
    \[
    \mathbf{V} = \begin{bmatrix} 1 \\ 1 \\ 0 \end{bmatrix}, \quad \alpha = 90^\circ
    \]
\end{enumerate}



\subsection*{9. Zbuduj macierz obrotu \( \mathbf{R} \) pozwalającą wykonać operację obrotu o kąt \( \alpha \) wokół osi \( \mathbf{Z} \).}

Macierz obrotu wokół osi \( Z \) ma postać:


\begin{enumerate}
    \item[(a)] \( \alpha = 30^\circ \)
    \item[(b)] \( \alpha = 45^\circ \)
    \item[(c)] \( \alpha = 60^\circ \)
    \item[(d)] \( \alpha = 90^\circ \)
\end{enumerate}



\subsection*{10. Pracując we współrzędnych jednorodnych zbuduj macierz transformacji \( \mathbf{F} \) pozwalającą wykonać operację skalowania \( \mathbf{S} \) oraz translacji \( \mathbf{T} \) jednocześnie. 
Oblicz pozycję punktów \( P_1', P_2', P_3' \), które powstaną po zastosowaniu do nich macierzy transformacji \( \mathbf{F} \).}

\begin{enumerate}
    \item[(a)] 
    \[
    P_1 = \begin{bmatrix} 0 \\ 0 \\ 0  \end{bmatrix}, 
    \quad P_2 = \begin{bmatrix} 1 \\ 1 \\ 1 \end{bmatrix}, 
    \quad P_3 = \begin{bmatrix} 2 \\ 5 \\ 1  \end{bmatrix}
    \]

    \[
    S = \begin{bmatrix} 2 & 0 & 0 \\ 0 & 2 & 0 \\ 0 & 0 & 2 \end{bmatrix}, 
    \quad T = \begin{bmatrix} 0 \\ 0 \\ 0 \end{bmatrix}
    \]

    \item[(b)] 
    \[
    P_1 = \begin{bmatrix} 4 \\ 2 \\ 0  \end{bmatrix}, 
    \quad P_2 = \begin{bmatrix} 1 \\ 2 \\ 3  \end{bmatrix}, 
    \quad P_3 = \begin{bmatrix} 2 \\ 5 \\ 1  \end{bmatrix}
    \]

    \[
    S = \begin{bmatrix} 1 & 0 & 0 \\ 0 & 1 & 0 \\ 0 & 0 & 1 \end{bmatrix}, 
    \quad T = \begin{bmatrix} 1 \\ 1 \\ 1 \end{bmatrix}
    \]

    \item[(c)] 
    \[
    P_1 = \begin{bmatrix} 0 \\ 0 \\ 0  \end{bmatrix}, 
    \quad P_2 = \begin{bmatrix} 1 \\ 1 \\ 1  \end{bmatrix}, 
    \quad P_3 = \begin{bmatrix} 2 \\ 5 \\ 1  \end{bmatrix}
    \]

    \[
    S = \begin{bmatrix} 0 & 0 & 0 \\ 0 & 0 & 0 \\ 0 & 0 & 0 \end{bmatrix}, 
    \quad T = \begin{bmatrix} 1 \\ 1 \\ 1 \end{bmatrix}
    \]

    \item[(f)] 
    \[
    P_1 = \begin{bmatrix} 0 \\ 0 \\ 0  \end{bmatrix}, 
    \quad P_2 = \begin{bmatrix} 1 \\ 1 \\ 1  \end{bmatrix}, 
    \quad P_3 = \begin{bmatrix} 2 \\ 5 \\ 1  \end{bmatrix}
    \]

    \[
    S_x = 2, \quad S_y = 1, \quad S_z = 3, 
    \quad T = \begin{bmatrix} 1 \\ 1 \\ 2 \end{bmatrix}
    \]

\end{enumerate}


\subsection*{11. Zbuduj macierz transformacji \( \mathbf{M} \) pozwalającą wykonać kolejno operację obrotu o kąt \( \alpha \) wokół osi \( \mathbf{Z} \), obrotu o kąt \( \beta \) wokół osi \( \mathbf{Y} \) jednocześnie. 
Oblicz pozycje punktów \( P_1', P_2', P_3' \), które powstaną po zastosowaniu macierzy transformacji \( \mathbf{M} \) do punktów \( P_1, P_2, P_3 \).}

\begin{enumerate}
    \item[(a)] 
    \[
    P_1 = \begin{bmatrix} 0 \\ 0 \\ 0 \end{bmatrix}, 
    \quad P_2 = \begin{bmatrix} 1 \\ 1 \\ 1 \end{bmatrix}, 
    \quad P_3 = \begin{bmatrix} 2 \\ 5 \\ 1 \end{bmatrix}
    \]
    
    \[
    \quad \alpha = 45^\circ, \quad \beta = 90^\circ
    \]
\end{enumerate}

\subsection*{12. Pracując we współrzędnych jednorodnych zbuduj macierz transformacji \( \mathbf{F} \) 
pozwalającą wykonać kolejno operację skalowania \( \mathbf{S} \), obrotu o kąt \( \alpha \) wokół osi \( \mathbf{X} \), 
translacji \( \mathbf{T_1} \), obrotu o kąt \( \beta \) wokół osi \( \mathbf{Y} \) i translacji \( \mathbf{T_2} \) jednocześnie.
Oblicz pozycję punktów \( P_1', P_2', P_3' \), które powstaną po zastosowaniu macierzy transformacji \( \mathbf{F} \) 
do punktów \( P_1, P_2, P_3 \).}

\begin{enumerate}
    \item[(a)] 
    \[
    P_1 = \begin{bmatrix} 0 \\ 0 \\ 0 \end{bmatrix}, 
    \quad P_2 = \begin{bmatrix} 1 \\ 1 \\ 1 \end{bmatrix}, 
    \quad P_3 = \begin{bmatrix} 2 \\ 5 \\ 1 \end{bmatrix}
    \]

    \[
    S_x = 2, \quad S_y = 1, \quad S_z = 2, 
    \quad T_1 = \begin{bmatrix} 1 \\ 1 \\ 2 \end{bmatrix}, 
    \quad T_2 = \begin{bmatrix} 0 \\ 1 \\ 2 \end{bmatrix}
    \]

    \[
    \alpha = 45^\circ, \quad \beta = 90^\circ
    \]

    \item[(b)] 
    \[
    P_1 = \begin{bmatrix} 0 \\ 2 \\ 4 \end{bmatrix}, 
    \quad P_2 = \begin{bmatrix} 3 \\ 2 \\ 1 \end{bmatrix}, 
    \quad P_3 = \begin{bmatrix} 3 \\ -2 \\ 5 \end{bmatrix}
    \]

    \[
    S_x = 4, \quad S_y = 2, \quad S_z = 2, 
    \quad T_1 = \begin{bmatrix} 1 \\ -2 \\ 3 \end{bmatrix}, 
    \quad T_2 = \begin{bmatrix} 2 \\ 1 \\ 0 \end{bmatrix}
    \]

    \[
    \alpha = 45^\circ, \quad \beta = 120^\circ
    \]

    \item[(c)] 
    \[
    P_1 = \begin{bmatrix} 4 \\ 2 \\ 0 \end{bmatrix}, 
    \quad P_2 = \begin{bmatrix} 1 \\ 2 \\ 3 \end{bmatrix}, 
    \quad P_3 = \begin{bmatrix} -1 \\ 2 \\ -4 \end{bmatrix}
    \]

    \[
    S_x = 0.5, \quad S_y = 2, \quad S_z = 3, 
    \quad T_1 = \begin{bmatrix} 5 \\ -1 \\ 2 \end{bmatrix}, 
    \quad T_2 = \begin{bmatrix} 2 \\ 0 \\ 4 \end{bmatrix}
    \]

    \[
    \alpha = 150^\circ, \quad \beta = 270^\circ
    \]
\end{enumerate}

