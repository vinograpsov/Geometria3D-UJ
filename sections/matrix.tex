\section{Macierze}


\subsection*{1. Oblicz wyznaczniki dla poniższych macierzy i jeśli to możliwe wyznacz macierze odwrotne.}

\begin{enumerate}
    \item[(a)] \( M = \begin{bmatrix} 2 & 5 \\ 2 & 7 \end{bmatrix} \)
    
    \item[(b)] \( M = \begin{bmatrix} 1 & 8 & 9 \\ 0 & 4 & 3 \\ 5 & 0 & 5 \end{bmatrix} \)
    
    \item[(c)] \( M = \begin{bmatrix} 2 & 7 & 3 \\ 1 & 8 & 9 \\ 1 & 6 & 4 \end{bmatrix} \)
    
    \item[(d)] \( M = \begin{bmatrix} 0 & 2 & 4 & 3 \\ 5 & 0 & 2 & 5 \\ 5 & 1 & 4 & 2 \end{bmatrix} \)
    
    \item[(e)] \( M = \begin{bmatrix} 1 & 6 & 4 & 0 \\ 2 & 7 & 3 & 0 \\ 2 & 0 & 0 & 0 \\ 8 & 9 & 0 & 5 \end{bmatrix} \)
\end{enumerate}

\subsection*{2. Wyznacz macierze odwrotne używając metody Gaussa-Jordana.}

\begin{enumerate}
    \item[(a)] \( M = \begin{bmatrix} 1 & 1 & 2 \\ 1 & 2 & 4 \\ 1 & 6 & 4 \end{bmatrix} \)
    
    \item[(b)] \( M = \begin{bmatrix} 2 & 7 & 3 \\ 8 & 9 & 5 \end{bmatrix} \)
\end{enumerate}

\subsection*{3. Dokonaj diagonalizacji macierzy.}

\begin{enumerate}
    \item[(a)] \( M = \begin{bmatrix} 1 & 2 & 3 & 4 \\ 2 & 1 & 5 \\ 1 & 6 & 4 \end{bmatrix} \)
    
    \item[(b)] \( M = \begin{bmatrix} 2 & 7 & 3 \\ 8 & 9 & 5 \end{bmatrix} \)
\end{enumerate}

\subsection*{4. Znajdź wartości własne i wektory własne macierzy.}

\begin{enumerate}
    \item[(a)] \( M = \begin{bmatrix} 1 & 0.5 \\ 0 & 2 \end{bmatrix} \)
    
    \item[(b)] \( M = \begin{bmatrix} 5 & 3 \\ -6 & -4 \end{bmatrix} \)
    
    \item[(c)] \( M = \begin{bmatrix} 1 & 1 \\ 4 & 1 \end{bmatrix} \)
    
    \item[(d)] \( M = \begin{bmatrix} 2 & 1 & 0 \\ 1 & 2 & 0 \\ 0 & 1 & 1 \end{bmatrix} \)
    
    \item[(e)] \( M = \begin{bmatrix} 2 & 1 & 0 \\ -6 & 1 & -6 \\ -3 & 1 & -1 \end{bmatrix} \)
    
    \item[(f)] \( M = \begin{bmatrix} 3 & 0 & 0 \\ -1 & 0 & 4 \\ 2 & 1 & 0 \end{bmatrix} \)
\end{enumerate}

\subsection*{5. Wykonaj operacje skalowania macierzą \( M \) dla macierzy wierzchołków \( N \).}

\begin{enumerate}
    \item[(a)] \( M = \begin{bmatrix} 1 & 0 & 0 \\ 0 & 2 & 0 \\ 0 & 0 & 3 \end{bmatrix}, \quad
    N = \begin{bmatrix} 1 & 1 & 3 \\ -1 & 2 \\ \end{bmatrix} \)
\end{enumerate}

\subsection*{6. Zbuduj macierz skalowania \( S \) dla której wynikiem operacji na punkcie \( P_1 \) będzie punkt \( P_2 \).}

\begin{enumerate}
    \item[(a)] \( P_1 = \begin{bmatrix} 2 \\ 4 \end{bmatrix}, \quad
    P_2 = \begin{bmatrix} 0 \\ 2 \end{bmatrix} \)
\end{enumerate}

\subsection*{7. Siatka geometryczna obiektu zawiera punkty \( P_1, P_2, P_3, P_4 \). Oblicz pozycję punktów \( P_1', P_2', P_3', P_4' \), które powstaną po zastosowaniu do nich macierzy skalowania \( S \) względem punktu \( P_T \).}

\begin{enumerate}
    \item[(a)] \( S = \begin{bmatrix} 2 & 0 & 0 \\ 0 & 2 & 0 \\ 0 & 0 & 1 \end{bmatrix}, \quad
    P_T = \begin{bmatrix} 1 \\ 1 \\ 0 \end{bmatrix} \)
\end{enumerate}

\subsection*{8. Wyznacz wektor \( V' \) będący wektorem powstałym w wyniku obrotu wektora \( V \) o kąt \( \alpha \) w osi \( Z \).}

\begin{enumerate}
    \item[(a)] \( V = \begin{bmatrix} 3 \\ 9 \\ 0 \end{bmatrix}, \quad \alpha = 60^\circ \)
\end{enumerate}

\subsection*{9. Zbuduj macierz obrotu \( R \) pozwalającą wykonać operację obrotu o kąt \( \alpha \) wokół osi \( Z \).}

\begin{enumerate}
    \item[(a)] \( \alpha = 30^\circ \)
\end{enumerate}

\subsection*{10. Pracując we współrzędnych jednorodnych zbuduj macierz transformacji \( F \) pozwalającą wykonać operację skalowania \( S \) oraz translacji \( T \) jednocześnie.}

\begin{enumerate}
    \item[(a)] \( S = \begin{bmatrix} 2 & 0 & 0 \\ 0 & 2 & 0 \\ 0 & 0 & 1 \end{bmatrix}, \quad
    T = \begin{bmatrix} 1 \\ 2 \\ 0 \end{bmatrix} \)
\end{enumerate}

\subsection*{11. Zbuduj macierz transformacji \( M \) pozwalającą wykonać kolejno operacje obrotu o kąt \( \alpha \) wokół osi \( Z \), obrotu o kąt \( \beta \) wokół osi \( Y \) jednocześnie.}

\begin{enumerate}
    \item[(a)] \( \alpha = 45^\circ, \quad \beta = 90^\circ \)
\end{enumerate}