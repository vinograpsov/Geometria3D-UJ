\section{Formulas}


Długość wektora \( \mathbf{V} \) jest określona jako:
\[
\|\mathbf{V}\| = \sqrt{V_x^2 + V_y^2 + V_z^2}
\]
\textbf{Opis:} Ten wzór pozwala obliczyć długość (moduł) wektora w przestrzeni trójwymiarowej, używając jego współrzędnych \( V_x \), \( V_y \), \( V_z \).

\vspace{1em}
\noindent
Iloczyn skalarny dwóch wektorów \( \mathbf{V} \) i \( \mathbf{W} \) jest określony jako:

\[
\mathbf{V} \cdot \mathbf{W} = V_x W_x + V_y W_y + V_z W_z
\]
\noindent
\textbf{Opis:} Ten wzór oblicza iloczyn skalarny (dot product) dwóch wektorów w przestrzeni trójwymiarowej, mnożąc odpowiadające sobie współrzędne \( V_x \), \( W_x \), \( V_y \), \( W_y \), \( V_z \), \( W_z \) i sumując wyniki.


\vspace{1em}
\noindent
Iloczyn wektorowy dwóch wektorów \( \mathbf{V} \) i \( \mathbf{W} \) jest określony jako:

\[
\mathbf{V} \times \mathbf{W} = 
\begin{bmatrix}
V_y W_z - V_z W_y \\
V_z W_x - V_x W_z \\
V_x W_y - V_y W_x
\end{bmatrix}
\]

\vspace{1em}
\noindent
\textbf{Opis:} Iloczyn wektorowy (cross product) dwóch wektorów w przestrzeni trójwymiarowej tworzy nowy wektor, który jest prostopadły do płaszczyzny utworzonej przez \( \mathbf{V} \) i \( \mathbf{W} \). Składniki nowego wektora są obliczane na podstawie powyższego wzoru.



\vspace{1em}
\noindent
Rzut wektora \( \mathbf{W} \) na \( \mathbf{V} \) jest określony jako:

\[
\text{proj}_{\mathbf{V}} \mathbf{W} = \frac{\mathbf{V} \cdot \mathbf{W}}{\|\mathbf{V}\|^2} \mathbf{V}
\]

\vspace{1em}
\noindent
\textbf{Opis:} Rzut wektora \( \mathbf{W} \) na \( \mathbf{V} \) to wektor będący projekcją \( \mathbf{W} \) na kierunek \( \mathbf{V} \).

