\section{Formulas}


Długość wektora \( \mathbf{V} \) jest określona jako:
\[
\|\mathbf{V}\| = \sqrt{V_x^2 + V_y^2 + V_z^2}
\]
\textbf{Opis:} Ten wzór pozwala obliczyć długość (moduł) wektora w przestrzeni trójwymiarowej, używając jego współrzędnych \( V_x \), \( V_y \), \( V_z \).

\vspace{1em}
\noindent
Iloczyn skalarny dwóch wektorów \( \mathbf{V} \) i \( \mathbf{W} \) jest określony jako:

\[
\mathbf{V} \cdot \mathbf{W} = V_x W_x + V_y W_y + V_z W_z
\]
\noindent
\textbf{Opis:} Ten wzór oblicza iloczyn skalarny (dot product) dwóch wektorów w przestrzeni trójwymiarowej, mnożąc odpowiadające sobie współrzędne \( V_x \), \( W_x \), \( V_y \), \( W_y \), \( V_z \), \( W_z \) i sumując wyniki.


\vspace{1em}
\noindent
Iloczyn wektorowy dwóch wektorów \( \mathbf{V} \) i \( \mathbf{W} \) jest określony jako:

\[
\mathbf{V} \times \mathbf{W} = 
\begin{bmatrix}
V_y W_z - V_z W_y \\
V_z W_x - V_x W_z \\
V_x W_y - V_y W_x
\end{bmatrix}
\]

\vspace{1em}
\noindent
\textbf{Opis:} Iloczyn wektorowy (cross product) dwóch wektorów w przestrzeni trójwymiarowej tworzy nowy wektor, który jest prostopadły do płaszczyzny utworzonej przez \( \mathbf{V} \) i \( \mathbf{W} \). Składniki nowego wektora są obliczane na podstawie powyższego wzoru.



\vspace{1em}
\noindent
Rzut wektora \( \mathbf{W} \) na \( \mathbf{V} \) jest określony jako:

\[
\text{proj}_{\mathbf{V}} \mathbf{W} = \frac{\mathbf{V} \cdot \mathbf{W}}{\|\mathbf{V}\|^2} \mathbf{V}
\]

\vspace{1em}
\noindent
\textbf{Opis:} Rzut wektora \( \mathbf{W} \) na \( \mathbf{V} \) to wektor będący projekcją \( \mathbf{W} \) na kierunek \( \mathbf{V} \).




\vspace{1em}
\noindent
Każdy układ liniowo niezależnych wektorów \( \mathbf{U}, \mathbf{V} \) i \( \mathbf{W} \) można przekształcić w ortonormalny układ \( \mathbf{u}, \mathbf{v}, \mathbf{w} \) za pomocą algorytmu ortonormalizacji Grama-Schmidta:

\[
\mathbf{u} = \frac{\mathbf{U}}{\|\mathbf{U}\|}
\]

\[
\mathbf{v} = \frac{\mathbf{V} - \text{proj}_{\mathbf{u}} \mathbf{V}}{\|\mathbf{V} - \text{proj}_{\mathbf{u}} \mathbf{V}\|}
\]

\[
\mathbf{w} = \frac{\mathbf{W} - \text{proj}_{\mathbf{u}} \mathbf{W} - \text{proj}_{\mathbf{v}} \mathbf{W}}{\|\mathbf{W} - \text{proj}_{\mathbf{u}} \mathbf{W} - \text{proj}_{\mathbf{v}} \mathbf{W}\|}
\]

\vspace{1em}
\noindent
\textbf{Opis:} Algorytm ortonormalizacji Grama-Schmidta służy do przekształcenia układu liniowo niezależnych wektorów w układ ortonormalny (wektory wzajemnie prostopadłe o długości 1). Proces polega na:
- Normalizacji pierwszego wektora \( \mathbf{U} \),
- Usuwaniu rzutów kolejnych wektorów na wcześniejsze i normalizacji powstałych wyników.



% not confident about this 
\vspace{1em}
\noindent
Wyznacznik macierzy \( M - \lambda I \) jest obliczany jako:

\[
\text{det}(M - \lambda I) = 
\begin{vmatrix}
1 - \lambda & 0 & 1 \\
0 & 1 - \lambda & 0 \\
1 & 0 & 1 - \lambda
\end{vmatrix}
\]

\vspace{1em}
\noindent
\textbf{Opis:} Wzór ten przedstawia wyznacznik macierzy \( M - \lambda I \), gdzie \( \lambda \) to wartość własna (eigenvalue), a \( I \) to macierz jednostkowa. Wyznacznik jest wykorzystywany w obliczaniu wartości własnych macierzy oraz w analizie jej własności algebraicznych.



\vspace{1em}
\noindent
Wyznacznik macierzy \( 3 \times 3 \) można rozwinąć wzdłuż pierwszego wiersza w następujący sposób:

% need to make better formatting
\begin{multline*}
    \begin{vmatrix}
    M_{xx} & M_{xy} & M_{xz} \\
    M_{yx} & M_{yy} & M_{yz} \\
    M_{zx} & M_{zy} & M_{zz}
    \end{vmatrix}
    =
    M_{xx}(-1)^{1+1}
    \begin{vmatrix}
    M_{yy} & M_{yz} \\
    M_{zy} & M_{zz}
    \end{vmatrix} \\
    + M_{xy}(-1)^{1+2}
    \begin{vmatrix}
    M_{yx} & M_{yz} \\
    M_{zx} & M_{zz}
    \end{vmatrix}
    + M_{xz}(-1)^{1+3}
    \begin{vmatrix}
    M_{yx} & M_{yy} \\
    M_{zx} & M_{zy}
    \end{vmatrix}
    \end{multline*}
    
    \begin{align*}
    &= M_{xx}M_{yy}M_{zz} - M_{xx}M_{yz}M_{zy} \\
    &\quad - M_{xy}M_{yx}M_{zz} + M_{xy}M_{yz}M_{zx} \\
    &\quad + M_{xz}M_{yx}M_{zy} - M_{xz}M_{yy}M_{zx}
    \end{align*}

\vspace{1em}
\noindent
\textbf{Opis:} Wyznacznik macierzy \( 3 \times 3 \) oblicza się, rozwijając go wzdłuż wiersza lub kolumny. Wzór ten przedstawia rozwinięcie wzdłuż pierwszego wiersza z zastosowaniem wyznaczników macierzy \( 2 \times 2 \) dla odpowiednich podmacierzy.




% need to make better formatting
\vspace{1em}
\noindent
Iloczyn dwóch macierzy \( M \) i \( K \) można zapisać jako:

\[
\resizebox{\textwidth}{!}{$
MK =
\begin{pmatrix}
M_{xx}K_{xx} + M_{xy}K_{yx} + M_{xz}K_{zx} & M_{xx}K_{xy} + M_{xy}K_{yy} + M_{xz}K_{zy} & M_{xx}K_{xz} + M_{xy}K_{yz} + M_{xz}K_{zz} \\
M_{yx}K_{xx} + M_{yy}K_{yx} + M_{yz}K_{zx} & M_{yx}K_{xy} + M_{yy}K_{yy} + M_{yz}K_{zy} & M_{yx}K_{xz} + M_{yy}K_{yz} + M_{yz}K_{zz} \\
M_{zx}K_{xx} + M_{zy}K_{yx} + M_{zz}K_{zx} & M_{zx}K_{xy} + M_{zy}K_{yy} + M_{zz}K_{zy} & M_{zx}K_{xz} + M_{zy}K_{yz} + M_{zz}K_{zz}
\end{pmatrix}
$}
\]


\vspace{1em}
\noindent
\textbf{Opis:} Iloczyn dwóch macierzy \( M \) i \( K \) oblicza się jako sumę iloczynów odpowiednich wierszy macierzy \( M \) i kolumn macierzy \( K \). Wynikiem jest nowa macierz, której elementy są obliczane zgodnie z powyższym wzorem.


