\section{Frustum}

\subsection*{1. Wyznacz płaszczyzny bryły widzenia}

Podane są parametry kąt widzenia \( \beta \), rozdzielczość ekranu oraz zakres rysowania \( n, f \):

\begin{enumerate}
    \item[(a)] \( \beta = 90^\circ, \quad 1280 \times 960, \quad n = 1, \quad f = 100 \)
    \item[(b)] \( \beta = 45^\circ, \quad 1920 \times 1080, \quad n = 1, \quad f = 100 \)
    \item[(c)] \( \beta = 60^\circ, \quad 1024 \times 768, \quad n = 1, \quad f = 100 \)
\end{enumerate}

\subsection*{2. Sprawdź, czy punkt \( P \) leży wewnątrz bryły widzenia}

\begin{enumerate}
    \item[(a)] \( P = \begin{bmatrix} 0 \\ 0 \\ -10 \end{bmatrix}, \quad a = 4:3, \quad e = 1, \quad n = 1, \quad f = 100 \)
    \item[(b)] \( P = \begin{bmatrix} 10 \\ 10 \\ -10 \end{bmatrix}, \quad a = 16:9, \quad e = 1, \quad n = 1, \quad f = 100 \)
    \item[(c)] \( P = \begin{bmatrix} 15 \\ 15 \\ 10 \end{bmatrix}, \quad a = 16:10, \quad e = 2, \quad n = 1, \quad f = 100 \)
\end{enumerate}

\subsection*{3. Wyznacz wektor normalny \( N \) i określ relacje punktu \( O \)}

Dane są płaszczyzny \( A, B, C \) zdefiniowane macierzą punktów:

\begin{enumerate}
    \item[(a)] 
    \[
    A = \begin{bmatrix} 0 & 5 & 0 \\ 0 & 0 & 3 \\ 0 & 0 & 0 \end{bmatrix}, \quad
    B = \begin{bmatrix} 0 & 0 & 5 \\ -1 & -1 & -1 \\ 0 & 6 & 0 \end{bmatrix}, \quad
    C = \begin{bmatrix} 1 & 1 & 1 \\ 0 & 0 & 3 \\ 0 & 9 & 0 \end{bmatrix}
    \]
    
    \[
    O = \begin{bmatrix} 2 \\ 1 \\ 3 \end{bmatrix}
    \]

    \item[(b)] 
    \[
    A = \begin{bmatrix} 0 & 15 & 0 \\ 0 & 0 & 3 \\ 1 & 1 & 1 \end{bmatrix}, \quad
    B = \begin{bmatrix} 0 & 0 & 5 \\ -1 & -1 & -1 \\ 0 & 6 & 0 \end{bmatrix}, \quad
    C = \begin{bmatrix} 1 & 1 & 1 \\ 0 & 0 & 3 \\ 0 & 9 & 0 \end{bmatrix}
    \]
    
    \[
    O = \begin{bmatrix} 3 \\ 2 \\ 1 \end{bmatrix}
    \]

    \item[(c)] 
    \[
    A = \begin{bmatrix} 1 & 2 & 2 \\ 1 & 2 & 2 \\ 1 & 1 & 2 \end{bmatrix}, \quad
    B = \begin{bmatrix} 0 & 2 & 4 \\ 0 & 2 & 2 \\ 0 & 0 & 0 \end{bmatrix}, \quad
    C = \begin{bmatrix} -1 & -1 & 0 \\ -1 & -1 & 0 \\ -1 & 0 & 0 \end{bmatrix}
    \]
    
    \[
    O = \begin{bmatrix} -1 \\ -1 \\ -1 \end{bmatrix}
    \]

\end{enumerate}
\subsection*{4. Sprawdź czy punkty \( P_1 \) i \( P_2 \) leżą wewnątrz bryły widzenia (frustum) zdefiniowanej płaszczyznami:}

\[
\forall i \in \{1,2,\dots,6\}
:\quad
n_{i,x}\,x \;+\; n_{i,y}\,y \;+\; n_{i,z}\,z \;+\; d_i \;\;\ge\; 0.
\]

\text{(Jeśli powyższy warunek jest spełniony dla każdego }i,\text{ punkt jest wewnątrz frustum.)}

\[
\begin{array}{c|cccccc}
    & \text{bliska} & \text{daleka} & \text{lewa} & \text{prawa} & \text{górna} & \text{dolna} \\
    \hline
    & 0 & 0 & \sqrt{2}/2 & -\sqrt{2}/2 & 0 & 0 \\
    n & 0 & 0 & 0 & 0 & \sqrt{2}/2 & -\sqrt{2}/2 \\
    a & -1 & 1 & -\sqrt{2}/2 & -\sqrt{2}/2 & \sqrt{2}/2 & -\sqrt{2}/2 \\
    & -1 & 100 & 0 & 0 & 0 & 0 \\
\end{array}
\]

\[
P_1 = \begin{bmatrix} 10 \\ 0 \\ -10 \end{bmatrix}, \quad
P_2 = \begin{bmatrix} 0 \\ 10 \\ -5 \end{bmatrix}
\]


\subsection*{5. Sprawdź czy sfery w punktach \( P_1 \) i \( P_2 \), i promieniach \( r_1 \) i \( r_2 \) leżą wewnątrz bryły widzenia (frustum) zdefiniowanej płaszczyznami:}

\[
\forall i \in \{1,2,\dots,6\}
:\quad
n_{i,x}\,x \;+\; n_{i,y}\,y \;+\; n_{i,z}\,z \;+\; d_i \;\;\ge\; r.
\]


\[
\begin{array}{c|cccccc}
    & \text{bliska} & \text{daleka} & \text{lewa} & \text{prawa} & \text{górna} & \text{dolna} \\
    \hline
    N & 0 & 0 & \sqrt{2}/2 & -\sqrt{2}/2 & 0 & 0 \\
    & 0 & 0 & 0 & 0 & \sqrt{2}/2 & -\sqrt{2}/2 \\
    a & -1 & 1 & -\sqrt{2}/2 & -\sqrt{2}/2 & \sqrt{2}/2 & -\sqrt{2}/2 \\
    & -1 & 100 & 0 & 0 & 0 & 0 \\
\end{array}
\]

\[
P_1 = \begin{bmatrix} 10 \\ 0 \\ -10 \end{bmatrix}, \quad r_1 = 3
\]

\[
P_2 = \begin{bmatrix} 0 \\ 10 \\ -5 \end{bmatrix}, \quad r_2 = 2
\]


